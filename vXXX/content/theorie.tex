\section{Zielsetzung}\label{sec:ziel}

\section{Theoretische Grundlagen}\label{sec:Theorie}
\subsection{Berechnung der Messunsicherheiten}\label{ssec:Fehlerrechnung}
Alle Mittelwerte einer $N$-fach gemessenen Größe $x$ werden über die Formel
\begin{equation}
    \overline{x} = \frac{1}{N} \sum_{i=1}^{N} x_\text{i} .
    \label{eqn:mittel}
\end{equation}
berechnet. Der zugehörige Fehler des Messwertes berechnet sich dann über
\begin{equation}
    \symup{\Delta} \overline{x} = \sqrt{\frac{1}{N(N-1)} \sum_{i=1}^{N} {(x_\text{i}-\overline{x})}^2} .
    \label{eqn:std}
\end{equation}
Setzt sich eine zu berechnende Größe aus mehreren mit Unsicherheit behafteten Messwerten zusammen, so ist die Unsicherheit dieser Größe über die Gaußsche Fehlerfortpflanzung gegeben
\begin{equation}
    \symup{\Delta} f(x_1, \ldots, x_\text{N}) = \sqrt{\sum_{i=1}^{N} \left[ {\left(\frac{\partial f}{\partial x_\text{i}}\right)}^2 \cdot {(\symup{\Delta} x_\text{i})}^2 \right] } .
    \label{eqn:gaus}
\end{equation}
Ausgleichsgraden lassen sich wie folgt berechnen:
\begin{subequations}
    \begin{equation}
        y = m \cdot x + b
        \label{eqn:grade}
    \end{equation}
    \begin{equation}
        m = \frac{\overline{xy} - \overline{x} \cdot \overline{y}}{\overline{x^2} - \overline{x}^2}
        \label{eqn:steigung}
    \end{equation}
    \begin{equation}
        b = \frac{\overline{x^2} \cdot \overline{y} - \overline{x} \cdot \overline{xy}}{\overline{x^2} - \overline{x}^2} .
        \label{eqn:achsenab}
    \end{equation}
\end{subequations}
Bei der Angabe des Endergebnisses werden schließlich alle statistischen Teilfehler addiert.
Alle Berechnungen, Graphen sowie das Bestimmen der Unsicherheiten werden mit Python 3.8.8 und entsprechenden Bibliotheken\footnote{Numpy~\cite{numpy}, Uncertainties~\cite{uncertainties} and Matplotlib~\cite{matplotlib}} durchgeführt.
