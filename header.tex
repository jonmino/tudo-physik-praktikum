\documentclass[
  bibliography=totoc,     % Literatur im Inhaltsverzeichnis
  captions=tableheading,  % Tabellenüberschriften
  titlepage=firstiscover, % Titelseite ist Deckblatt
]{scrartcl}

% Paket float verbessern
\usepackage{scrhack}

% Warnung, falls nochmal kompiliert werden muss
\usepackage[aux]{rerunfilecheck}

% unverzichtbare Mathe-Befehle
\usepackage{amsmath}
% viele Mathe-Symbole
\usepackage{amssymb}
% Erweiterungen für amsmath
\usepackage{mathtools}

% Fonteinstellungen
\usepackage{fontspec}

% deutsche Spracheinstellungen
\usepackage[ngerman]{babel}

\usepackage[
  math-style=ISO,    % ┐
  bold-style=ISO,    % │
  sans-style=italic, % │ ISO-Standard folgen
  nabla=upright,     % │
  partial=upright,   % ┘
  warnings-off={             % ┐
      mathtools-colon,       % │ unnötige Warnungen ausschalten
      mathtools-overbracket, % │
    },                       % ┘
]{unicode-math}

% Zahlen und Einheiten
\usepackage[
  locale=DE,                   % deutsche Einstellungen
  separate-uncertainty=true,   % immer Unsicherheit mit \pm
  per-mode=symbol-or-fraction, % / in inline math, fraction in display math
  table-align-uncertainty=true,% richtige Ausrichtung bei Tabellen
  table-align-exponent=true,   % Ausrichtung des Exponenten
  round-mode=figures,      % Werte mit Unsicherheit (u) runden, diese auf round-precision, options: uncertainty, none, places, figures
  round-precision=2,           % auf zwei Stellen
  tight-spacing=true,     % Abstand zwischen Zahl und Einheit
  table-alignment-mode=format, % Use text-format as Space
]{siunitx}
\DeclareSIUnit\bar{bar}

% chemische Formeln
\usepackage[
  version=4,
  math-greek=default, % ┐ mit unicode-math zusammenarbeiten
  text-greek=default, % ┘
]{mhchem}

%Font Definition
%\setmathfont{Libertinus Math}
%\setromanfont{Libertinus Serif}
%\setsansfont{Libertinus Sans}
%\setmonofont{Libertinus Mono}

% LaTeX
\usepackage{FiraSans}
% \setmainfont{STIX Two Text}
% \setmathfont{XITS Math}
\setmainfont{FiraSans Regular}[
  Scale=MatchLowercase,
  BoldFont=FiraSans SemiBold,
]
\setsansfont{FiraSans Light}[
  Scale=MatchLowercase,
  BoldFont=FiraSans Medium,
]
\setmonofont{Fira Mono}[Scale=MatchLowercase]
\setmathfont{Fira Math}[Scale=MatchLowercase]
%\setmathfont{STIX Two Math}

% Wenn man andere Schriftarten gesetzt hat,
% sollte man das Seiten-Layout neu berechnen lassen
\recalctypearea{}

% richtige Anführungszeichen
\usepackage[autostyle]{csquotes}

% schöne Brüche im Text
\usepackage{xfrac}

% Standardplatzierung für Floats einstellen
\usepackage{float}
\floatplacement{figure}{htbp}
\floatplacement{table}{htbp}

% Floats innerhalb einer Section halten
\usepackage[
  section, % Floats innerhalb der Section halten
  below,   % unterhalb der Section aber auf der selben Seite ist ok
]{placeins}

% Seite drehen für breite Tabellen: landscape Umgebung
\usepackage{pdflscape}

% Captions schöner machen.
\usepackage[
  labelfont=bf,        % Tabelle x: Abbildung y: ist jetzt fett
  font=small,          % Schrift etwas kleiner als Dokument
  width=0.9\textwidth, % maximale Breite einer Caption schmaler
]{caption}
% subfigure, subtable, subref
\usepackage{subcaption}

% Grafiken können eingebunden werden
\usepackage{graphicx}
\usepackage{wrapfig}%Textumflossene Graphik
\usepackage{cancel} %Brüche Kürzen
\usepackage{tikz} %Fancy Kreisnummern
\usepackage[a4paper, left=25mm, top=25mm, right=25mm, bottom=25mm]{geometry} %Pagelayout
\usepackage{subcaption} %Unterüberschrift
\captionsetup[figure]{calcwidth=.85\linewidth} %You dont want to know
\usepackage{longtable}

% schöne Tabellen
\usepackage{booktabs}

% Verbesserungen am Schriftbild
\usepackage{microtype}

% Literaturverzeichnis
\usepackage[backend=biber,style=numeric,sorting=none,backrefstyle=two]{biblatex}
% Quellendatenbank
\addbibresource{lit.bib}
\addbibresource{programme.bib}

% Hyperlinks im Dokument
\usepackage[
  german,
  unicode,        % Unicode in PDF-Attributen erlauben
  pdfusetitle,    % Titel, Autoren und Datum als PDF-Attribute
  pdfcreator={},  % ┐ PDF-Attribute säubern
  pdfproducer={}, % ┘
]{hyperref}
% erweiterte Bookmarks im PDF
\usepackage{bookmark}

% Trennung von Wörtern mit Strichen
\usepackage[shortcuts]{extdash}

\usepackage[headsepline=1pt,footsepline=1pt]{scrlayer-scrpage}
\pagestyle{scrheadings}
\clearpairofpagestyles

\author{%
  Irgendjemand\\%
  \href{mailto:irgend.jemand@tu-dortmund.de}{irgend.jemand@tu-dortmund.de}%
  \and%
  Someone\\%
  \href{mailto:some.one@tu-dortmund.de}{some.one@tu-dortmund.de}%
}
\publishers{TU Dortmund – Fakultät Physik}
\setlength\parindent{0pt}
